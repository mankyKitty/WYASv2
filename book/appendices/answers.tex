\chapter{Answers to Exercises}
 
\chapterlinks{Answers}
 
\section{Section 2.3}
\editsection{Answers}{1}
 
\subsection{Exercise 1}
\editsection{Answers}{2}
 
\subsubsection{Part 1}
\editsection{Answers}{3}
 
\begin{lstlisting}
parseNumber :: Parser LispVal
parseNumber = do x <- many1 digit
                (return . Number . read) x
\end{lstlisting}
 
\subsubsection{Part 2}
\editsection{Answers}{4}
 
In order to anwer this question, you need to do a bit of detective work! It is helpful to read up on do notation. Using the information there, we can mechanically transform the above answer into the following.
 
\begin{lstlisting}
parseNumber = many1 digit >>= \x -> (return . Number . read) x
\end{lstlisting}
 
This can be cleaned up into the following:
 
\begin{lstlisting}
parseNumber = many1 digit >>= return . Number . read
\end{lstlisting}
 
\subsection{Exercise 2}
\editsection{Answers}{5}
 
We need to create a new parser action that accepts a backslash followed by either another backslash or a doublequote. This action needs to return only the second character.
 
\begin{lstlisting}
escapedChars :: Parser String
escapedChars = do char '\\' -- a backslash
                  x <- oneOf "\\\"" -- either backslash or doublequote
                  return [x] -- make this character into a string
\end{lstlisting}
 
Once that is done, we need to make some changes to parseString.
 
\begin{lstlisting}
parseString :: Parser LispVal
parseString = do char '"'
                 x <- many $ many1 (noneOf "\"\\") <|> escapedChars
                 char '"'
                 return $ String (concat x)
\end{lstlisting}
 
\subsection{Exercise 3}
\editsection{Answers}{6}
 
\begin{lstlisting}
escapedChars :: Parser String
escapedChars = do char '\\' 
                  x <- oneOf "\\\"ntr" 
                  case x of 
                    '\\' -> do return [x]
                    '"' -> do return [x]
                    't' -> do return "\t"
                    'n' -> do return "\n"
                    'r' -> do return "\r"
\end{lstlisting}
 
\subsection{Exercise 4}
\editsection{Answers}{7}
 
First, it is necessary to change the definition of symbol.
 
\begin{lstlisting}
symbol :: Parser Char
symbol = oneOf "!$%&|*+-/:<;\color{green}=;>?@^_~"
\end{lstlisting}
 
This means that it is no longer possible to begin an atom with the hash character. This necessitates a different way of parsing \verb|#t| and \verb|#f|.
 
\begin{lstlisting}
parseBool :: Parser LispVal
parseBool = do string "#"
               x <- oneOf "tf"
               return $ case x of 
                          't' -> Bool True
                          'f' -> Bool False
\end{lstlisting}
 
This in turn requires us to make changes to parseExpr.
 
\begin{lstlisting}
parseExpr :: Parser LispVal
parseExpr = parseAtom
        <|> parseString
        <|> parseNumber
        <|> parseBool
\end{lstlisting}
 
\verb|parseNumber| need to be changed to the following.
 
\begin{lstlisting}
parseNumber :: Parser LispVal
parseNumber = do num <- parseDigital1 <|> parseDigital2 <|>
        parseHex <|> parseOct <|> parseBin
                 return $ num
\end{lstlisting}
 
And the following new functions need to be added.
 
 
\begin{lstlisting}
parseDigital1 :: Parser LispVal
parseDigital1 = do x <- many1 digit
                   (return . Number . read) x   
\end{lstlisting}
 
\begin{lstlisting}
parseDigital2 :: Parser LispVal
parseDigital2 = do try $ string "#d"
                   x <- many1 digit
                   (return . Number . read) x
\end{lstlisting}
 
\begin{lstlisting}
parseHex :: Parser LispVal
parseHex = do try $ string "#x"
              x <- many1 hexDigit
               return $ Number (hex2dig x)
\end{lstlisting}
 
\begin{lstlisting}
parseOct :: Parser LispVal
parseOct = do try $ string "#o"
              x <- many1 octDigit
              return $ Number (oct2dig x)
\end{lstlisting}
 
\begin{lstlisting}
parseBin :: Parser LispVal
parseBin = do try $ string "#b"
              x <- many1 (oneOf "10")
              return $ Number (bin2dig x)
\end{lstlisting}
 
\begin{lstlisting}
oct2dig x = fst $ readOct x !! 0
hex2dig x = fst $ readHex x !! 0
bin2dig  = bin2dig' 0
bin2dig' digint "" = digint
bin2dig' digint (x:xs) = let old = 2 * digint + (if x == '0' then 0 else 1) in bin2dig' old xs
\end{lstlisting}
 
\subsection{Exercise 5}
\editsection{Answers}{8}
 
\begin{lstlisting}
data LispVal = Atom String
             | List [LispVal]
             | DottedList [LispVal] LispVal
             | Number Integer
             | String String
             | Bool Bool
             | Character Char
\end{lstlisting}
 
\begin{lstlisting}
parseChar :: Parser LispVal
parseChar = do try $ string "#\\"
               x <- parseCharName <|> anyChar
               return $ Character x 
\end{lstlisting}
 
\begin{lstlisting}
parseCharName = do x <- try (string "space" <|> string "newline")
                   case x of 
                     "space" -> do return ' '
                     "newline" -> do return '\n'
\end{lstlisting}
 
Note that this does not actually conform to the standard; as it stands, \verb|space| and \verb|newline| must be entirely lowercase; the standard states that they should be case insensitive.
 
\begin{lstlisting}
parseExpr :: Parser LispVal
parseExpr = parseAtom
        <|> parseString
        <|> try parseNumber -- we need the 'try' because 
        <|> try parseBool -- these can all start with the hash char
        <|> try parseChar
\end{lstlisting}
